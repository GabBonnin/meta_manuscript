\documentclass[
  man,
  longtable,
  a4paper,
  nolmodern,
  notxfonts,
  notimes,
  colorlinks=true,linkcolor=blue,citecolor=blue,urlcolor=blue]{apa7}

\usepackage{amsmath}
\usepackage{amssymb}




\RequirePackage{longtable}
\RequirePackage{threeparttablex}

\makeatletter
\renewcommand{\paragraph}{\@startsection{paragraph}{4}{\parindent}%
	{0\baselineskip \@plus 0.2ex \@minus 0.2ex}%
	{-.5em}%
	{\normalfont\normalsize\bfseries\typesectitle}}

\renewcommand{\subparagraph}[1]{\@startsection{subparagraph}{5}{0.5em}%
	{0\baselineskip \@plus 0.2ex \@minus 0.2ex}%
	{-\z@\relax}%
	{\normalfont\normalsize\bfseries\itshape\hspace{\parindent}{#1}\textit{\addperi}}{\relax}}
\makeatother




\usepackage{longtable, booktabs, multirow, multicol, colortbl, hhline, caption, array, float, xpatch}
\usepackage{subcaption}


\renewcommand\thesubfigure{\Alph{subfigure}}
\setcounter{topnumber}{2}
\setcounter{bottomnumber}{2}
\setcounter{totalnumber}{4}
\renewcommand{\topfraction}{0.85}
\renewcommand{\bottomfraction}{0.85}
\renewcommand{\textfraction}{0.15}
\renewcommand{\floatpagefraction}{0.7}

\usepackage{tcolorbox}
\tcbuselibrary{listings,theorems, breakable, skins}
\usepackage{fontawesome5}

\definecolor{quarto-callout-color}{HTML}{909090}
\definecolor{quarto-callout-note-color}{HTML}{0758E5}
\definecolor{quarto-callout-important-color}{HTML}{CC1914}
\definecolor{quarto-callout-warning-color}{HTML}{EB9113}
\definecolor{quarto-callout-tip-color}{HTML}{00A047}
\definecolor{quarto-callout-caution-color}{HTML}{FC5300}
\definecolor{quarto-callout-color-frame}{HTML}{ACACAC}
\definecolor{quarto-callout-note-color-frame}{HTML}{4582EC}
\definecolor{quarto-callout-important-color-frame}{HTML}{D9534F}
\definecolor{quarto-callout-warning-color-frame}{HTML}{F0AD4E}
\definecolor{quarto-callout-tip-color-frame}{HTML}{02B875}
\definecolor{quarto-callout-caution-color-frame}{HTML}{FD7E14}

%\newlength\Oldarrayrulewidth
%\newlength\Oldtabcolsep


\usepackage{hyperref}




\providecommand{\tightlist}{%
  \setlength{\itemsep}{0pt}\setlength{\parskip}{0pt}}
\usepackage{longtable,booktabs,array}
\usepackage{calc} % for calculating minipage widths
% Correct order of tables after \paragraph or \subparagraph
\usepackage{etoolbox}
\makeatletter
\patchcmd\longtable{\par}{\if@noskipsec\mbox{}\fi\par}{}{}
\makeatother
% Allow footnotes in longtable head/foot
\IfFileExists{footnotehyper.sty}{\usepackage{footnotehyper}}{\usepackage{footnote}}
\makesavenoteenv{longtable}

\usepackage{graphicx}
\makeatletter
\newsavebox\pandoc@box
\newcommand*\pandocbounded[1]{% scales image to fit in text height/width
  \sbox\pandoc@box{#1}%
  \Gscale@div\@tempa{\textheight}{\dimexpr\ht\pandoc@box+\dp\pandoc@box\relax}%
  \Gscale@div\@tempb{\linewidth}{\wd\pandoc@box}%
  \ifdim\@tempb\p@<\@tempa\p@\let\@tempa\@tempb\fi% select the smaller of both
  \ifdim\@tempa\p@<\p@\scalebox{\@tempa}{\usebox\pandoc@box}%
  \else\usebox{\pandoc@box}%
  \fi%
}
% Set default figure placement to htbp
\def\fps@figure{htbp}
\makeatother


% definitions for citeproc citations
\NewDocumentCommand\citeproctext{}{}
\NewDocumentCommand\citeproc{mm}{%
  \begingroup\def\citeproctext{#2}\cite{#1}\endgroup}
\makeatletter
 % allow citations to break across lines
 \let\@cite@ofmt\@firstofone
 % avoid brackets around text for \cite:
 \def\@biblabel#1{}
 \def\@cite#1#2{{#1\if@tempswa , #2\fi}}
\makeatother
\newlength{\cslhangindent}
\setlength{\cslhangindent}{1.5em}
\newlength{\csllabelwidth}
\setlength{\csllabelwidth}{3em}
\newenvironment{CSLReferences}[2] % #1 hanging-indent, #2 entry-spacing
 {\begin{list}{}{%
  \setlength{\itemindent}{0pt}
  \setlength{\leftmargin}{0pt}
  \setlength{\parsep}{0pt}
  % turn on hanging indent if param 1 is 1
  \ifodd #1
   \setlength{\leftmargin}{\cslhangindent}
   \setlength{\itemindent}{-1\cslhangindent}
  \fi
  % set entry spacing
  \setlength{\itemsep}{#2\baselineskip}}}
 {\end{list}}
\usepackage{calc}
\newcommand{\CSLBlock}[1]{\hfill\break\parbox[t]{\linewidth}{\strut\ignorespaces#1\strut}}
\newcommand{\CSLLeftMargin}[1]{\parbox[t]{\csllabelwidth}{\strut#1\strut}}
\newcommand{\CSLRightInline}[1]{\parbox[t]{\linewidth - \csllabelwidth}{\strut#1\strut}}
\newcommand{\CSLIndent}[1]{\hspace{\cslhangindent}#1}





\usepackage{newtx}

\defaultfontfeatures{Scale=MatchLowercase}
\defaultfontfeatures[\rmfamily]{Ligatures=TeX,Scale=1}





\title{Mental Health Evaluation through Text Analysis (META)}


\shorttitle{Mental Health Evaluation through Text Analysis (META)}


\usepackage{etoolbox}






\author{Gabriel Bonnin}



\affiliation{
{Ruhr University Bochum, Germany}}




\leftheader{Bonnin}






\authornote{ 

\par{       }
\par{Correspondence concerning this article should be addressed
to Gabriel
Bonnin, Email: \href{mailto:gabriel.bonnin@ruhr-uni-bochum.de}{gabriel.bonnin@ruhr-uni-bochum.de}}
}

\makeatletter
\let\endoldlt\endlongtable
\def\endlongtable{
\hline
\endoldlt
}
\makeatother
\RequirePackage{longtable}
\DeclareDelayedFloatFlavor{longtable}{table}

\urlstyle{same}



\makeatletter
\@ifpackageloaded{caption}{}{\usepackage{caption}}
\AtBeginDocument{%
\ifdefined\contentsname
  \renewcommand*\contentsname{Table of contents}
\else
  \newcommand\contentsname{Table of contents}
\fi
\ifdefined\listfigurename
  \renewcommand*\listfigurename{List of Figures}
\else
  \newcommand\listfigurename{List of Figures}
\fi
\ifdefined\listtablename
  \renewcommand*\listtablename{List of Tables}
\else
  \newcommand\listtablename{List of Tables}
\fi
\ifdefined\figurename
  \renewcommand*\figurename{Figure}
\else
  \newcommand\figurename{Figure}
\fi
\ifdefined\tablename
  \renewcommand*\tablename{Table}
\else
  \newcommand\tablename{Table}
\fi
}
\@ifpackageloaded{float}{}{\usepackage{float}}
\floatstyle{ruled}
\@ifundefined{c@chapter}{\newfloat{codelisting}{h}{lop}}{\newfloat{codelisting}{h}{lop}[chapter]}
\floatname{codelisting}{Listing}
\newcommand*\listoflistings{\listof{codelisting}{List of Listings}}
\makeatother
\makeatletter
\makeatother
\makeatletter
\@ifpackageloaded{caption}{}{\usepackage{caption}}
\@ifpackageloaded{subcaption}{}{\usepackage{subcaption}}
\makeatother

% From https://tex.stackexchange.com/a/645996/211326
%%% apa7 doesn't want to add appendix section titles in the toc
%%% let's make it do it
\makeatletter
\xpatchcmd{\appendix}
  {\par}
  {\addcontentsline{toc}{section}{\@currentlabelname}\par}
  {}{}
\makeatother

%% Disable longtable counter
%% https://tex.stackexchange.com/a/248395/211326

\usepackage{etoolbox}

\makeatletter
\patchcmd{\LT@caption}
  {\bgroup}
  {\bgroup\global\LTpatch@captiontrue}
  {}{}
\patchcmd{\longtable}
  {\par}
  {\par\global\LTpatch@captionfalse}
  {}{}
\apptocmd{\endlongtable}
  {\ifLTpatch@caption\else\addtocounter{table}{-1}\fi}
  {}{}
\newif\ifLTpatch@caption
\makeatother

\begin{document}

\maketitle




\setlength\LTleft{0pt}




\section{Abstract}\label{abstract}

Psychotherapy is one of the most effective treatments for mental health
problems, but its success depends on accurate diagnostic assessments.
Current diagnostic tools often use standardized closed-ended scales
that, while reliable, may fail to capture the complexity and
individuality of mental states. In collaboration with Dr.~Oscar Kjell at
Lund University, this project leverages advances in artificial
intelligence (AI) and natural language processing (NLP) to transform the
way mental health is assessed.

This project is based on a unique longitudinal dataset collected over 10
years at the Mental Health Research and Treatment Center (FBZ) at Ruhr
University Bochum. It consists of written self-reports in which patients
describe their mental health problems, functional impairment, and
therapy goals in their own words. These texts are linked to key clinical
outcomes such as diagnoses, symptom severity, functional impairment, and
treatment success, providing a rich, ecologically valid resource for
understanding patient progress in psychotherapy.

While previous NLP-based mental health studies have focused on social
media posts, limiting their clinical applicability, this project applies
state-of-the-art large language models (LLMs) to real-world clinical
data. By analyzing patients' open-ended responses, we aim to uncover
patterns in how they articulate mental health, emotions, and treatment
trajectories. These insights will inform the development of AI-powered
tools that offer more personalized and clinically relevant assessments,
surpassing traditional methods in accuracy and depth. Ultimately, this
research aims to support clinicians in making more informed treatment
decisions, enhance personalized care, and contribute to the
modernization of mental health assessment.

\section{Introduction}\label{introduction}

Mental health problems pose a significant global challenge, accounting
for a considerable proportion of deaths and disability-adjusted life
years (\citeproc{ref-WorldHealthOrganization2017}{World Health
Organization, 2017}). Psychotherapy is an effective and sustainable
intervention for reducing symptoms and improving quality of life
(\citeproc{ref-chorpita2011evidence}{Chorpita et al., 2011};
\citeproc{ref-wampold2015great}{Wampold \& Imel, 2015}), but it's
success critically depends on accurate assessments
(\citeproc{ref-Jensen-Doss2008}{Jensen-Doss \& Weisz, 2008};
\citeproc{ref-Lutz2022}{Lutz et al., 2022}).

Current assessment practices typically combine subjective self-reports
with clinical observations (\citeproc{ref-Bonnin2024a}{Bonnin et al.,
2024}). Standardized closed-ended tools such as the Beck Depression
Inventory-II (\citeproc{ref-Beck1996}{Beck et al., 1996}) rely on
numerical scales (\citeproc{ref-Likert1932}{Likert, 1932}) to structure
and standardize assessments and are widely used in clinical research and
practice. While these methods have advanced replicability and
reliability in psychological assessment, they can miss important
individual differences by restricting responses to pre-defined
categories, limiting the ability to capture the complexity of mental
states (\citeproc{ref-Kjell2024}{Kjell, Kjell, et al., 2024}).

Recent advances in AI, particularly transformer-based LLMs
(\citeproc{ref-Vaswani2017}{Vaswani et al., 2017}), present promising
solutions to these limitations (\citeproc{ref-Kjell2024}{Kjell, Kjell,
et al., 2024}). LLMs excel in analyzing context-rich natural language
with remarkable accuracy across diverse tasks
(\citeproc{ref-devlin-etal-2019-bert}{Devlin et al., 2019}). Open-ended
response formats, where patients describe their experiences in their own
words, provide high-dimensional, context-rich information that remains
underutilized in current assessment practices. Empirical studies
highlight the potential of NLP-based analysis of open-ended responses,
achieving moderate convergence with closed-ended rating scales using
traditional NLP methods (\citeproc{ref-kjell2019semantic}{Kjell et al.,
2019}) and nearing theoretical upper limits of accuracy with LLMs
(\citeproc{ref-Kjell2022}{Kjell et al., 2022}). Preliminary research
also highlights their potential for predicting clinically significant
outcomes, including suicide risk
(\citeproc{ref-matero-etal-2019-suicide}{Matero et al., 2019};
\citeproc{ref-mohammadi-etal-2019-clac-clpsych}{Mohammadi et al., 2019};
\citeproc{ref-zirikly-etal-2019-clpsych}{Zirikly et al., 2019}).

At the FBZ at Ruhr University Bochum, open-ended patient responses have
been routinely collected from approximately 3,000 patients pre-therapy.
While previous studies have explored the use of NLP for mental health
assessment, they have largely relied on social media language, limiting
their clinical applicability. My project moves beyond the current
state-of-the-art by applying LLMs to analyze this unique, large-scale,
and longitudinal clinical dataset, assessing the relationship between
patients' probed mental health responses and clinically relevant
constructs such as diagnosis, symptom severity, and functional
impairment. Additionally, it will predict key clinical outcomes such as
treatment response and therapy goal attainment. Furthermore, the project
seeks to generate clinically meaningful, data-driven insights that go
beyond traditional diagnostic categories, offering a more nuanced and
patient-centered understanding of mental health trajectories.

\section{Methods}\label{methods}

\subsection{Preprocessing}\label{preprocessing}

To streamline data collection, an automated transcription pipeline was
implemented: The handwritten text data is first recorded by trained
employees of the FBZ adult outpatient clinic using a mobile audio
recording device. Identifying features (e.g.~names, dates of birth,
location details) are replaced by placeholders during recording
(anonymisation). The transcription is then carried out on local hardware
using the open source tool Whisper Large v2
(\url{https://github.com/openai/whisper}), a state-of-the-art
speech-to-text model (\citeproc{ref-Radford2022}{Radford et al., 2022}).
Each recording begins with a structured introduction, including a
patient identification code, followed by responses to predefined
questions. The transcription pipeline automatically processes all audio
recordings, extracts the patient codes, and identifies responses to key
questions. The data is then cleaned and structured into a tabular format
for further analysis.

\subsection{Measures}\label{measures}

\subsubsection{Sociodemographic and context
measures}\label{sociodemographic-and-context-measures}

demographics include age, sex, marital status, relationship status,
general education, vocational qualification, work ability

context factors include: Vorbehandlung, how therapy ended.

\subsubsection{Responses from open-ended questions before
therapy}\label{responses-from-open-ended-questions-before-therapy}

At the start of therapy, patients complete two separate questionnaires
designed to assess key aspects of their mental health concerns,
functional impairments, and expectations for treatment. Questions 1--9
come from the first questionnaire (\emph{Fragebogen zur
Lebensgeschichte}), and questions 10--13 come from the second
(\emph{Eingangsfragebogen}). The questions include:

\begin{enumerate}
\def\labelenumi{\arabic{enumi}.}
\tightlist
\item
  \textbf{Problem development:} `Briefly describe how the problems for
  which you are seeking treatment have developed over time.' (geman
  original question: „Beschreiben Sie kurz, wie sich Ihre Probleme,
  wegen derer Sie eine Behandlung aufsuchen, im Laufe der Zeit
  entwickelt haben.'')
\item
  \textbf{Extra stressors:} `What causes you stress in addition to your
  everyday problems (e.g.~finances, housing situation)?' (geman original
  question: „Was macht Ihnen zusätzlich zu Ihren Problemen im Alltag
  Stress (z. B. Finanzen, Wohnsituation)?{}``)
\item
  \textbf{Pre-onset changes:} `Did something special change in your life
  before the onset of your symptoms? (e.g.~death of an important person,
  divorce or separation, change in work situation or income, addition to
  the family)' (geman original question: „Hat sich vor dem Beginn Ihrer
  Beschwerden etwas Besonderes in Ihrem Leben verändert? (z. B. Tod
  einer wichtigen Bezugsperson, Scheidung oder Trennung, Veränderung der
  Arbeitssituation oder des Einkommens, Familienzuwachs)``)
\item
  \textbf{Event connection:} `Do you see a connection between the
  event(s) and the development of your problems?' (geman original
  question: „Sehen Sie einen Zusammenhang zwischen dem Ereignis/den
  Ereignissen und der Entwicklung Ihrer Probleme?{}``)
\item
  \textbf{Physical symptoms:} `Are there any physical side effects when
  your problems occur?' (geman original question: „Gibt es körperliche
  Begleiterscheinungen, wenn Ihre Probleme auftreten?{}``)
\item
  \textbf{Problem causes:} `What do you think are the causes of your
  problems?' (geman original question: „Welche Ursachen sehen Sie für
  Ihre Probleme?{}``)
\item
  \textbf{Expected improvements:} `What would improve in your life if
  you no longer had your problems?' (geman original question: „Was würde
  sich in Ihrem Leben verbessern, wenn Sie ihre Probleme nicht mehr
  hätten?{}``)
\item
  \textbf{Environment response:} `How does your environment (partner,
  family, friends, work colleagues) react to your problems?' (geman
  original question: „Wie reagiert Ihre Umwelt (Partner:in, Familie,
  Freund:innen, Arbeitskolleg:innen) auf die Probleme?{}``)
\item
  \textbf{No change required:} `What should not change under any
  circumstances as a result of the therapy?' (geman original question:
  „Was sollte sich durch die Therapie auf keinen Fall verändern?{}``)
\item
  \textbf{Problem description:} `Finally, please describe in your own
  words the problems for which you would like treatment.' (geman
  original question: „Beschreiben Sie zum Abschluss bitte noch einmal in
  eigenen Worten Ihre Probleme, deretwegen Sie eine Behandlung
  wünschen.``)
\item
  \textbf{Impacted life areas:} `In which areas of your life do these
  problems limit you (e.g.~job, relationship)?' (geman original
  question: „In welchen Lebensbereichen schränken Sie diese Probleme ein
  (z. B. Beruf, Partnerschaft)?{}``)
\item
  \textbf{Therapy goals:} `What would you like to achieve for yourself
  in therapy?' (geman original question: „Was möchten Sie in der
  Therapie für sich erreichen?{}``)
\end{enumerate}

\subsubsection{Psychometric measures}\label{psychometric-measures}

Clinical and psychometric variables were retrieved from the FBZ database
and included diagnostic information, self-report symptom measures,
therapist- and patient-rated outcome measures, positive mental health
indicators, and therapeutic process variables. Diagnoses were coded
according to DSM-5 and ICD-10 criteria. Symptom severity and treatment
outcomes were assessed using a combination of standardized self-report
questionnaires and clinician-rated instruments administered at different
points during treatment.

\paragraph{Diagnosis.}\label{diagnosis}

Diagnosis at the outpatient clinic is conducted using structured
clinical interviews. These typically take place before therapy begins,
usually at the fourth therapist--patient contact. The interview used is
the Diagnostic Interview for Mental Disorders
(\citeproc{ref-margraf2021}{Margraf et al., 2021}), which covers the
most frequent DSM-5 disorders encountered in outpatient therapy
settings.

\paragraph{Beck-Depression-Inventory
II.}\label{beck-depression-inventory-ii}

Depressive symptoms were assessed using the~\emph{Beck Depression
Inventory--II}~(BDI-II; (\citeproc{ref-Beck1996}{Beck et al., 1996})), a
widely used self-report questionnaire measuring the severity of
depressive symptomatology over the past two weeks.

\paragraph{Depression Anxiety Stress Scale
42.}\label{depression-anxiety-stress-scale-42}

General psychological distress was assessed using the \emph{Depression
Anxiety Stress Scales--42} (DASS-42(\citeproc{ref-Lovibond1995}{Lovibond
\& Lovibond, 1995})), which consists of 42 items measuring symptoms of
depression, anxiety, and stress on a XXX scale.

\paragraph{Brief Symptom Inventory.}\label{brief-symptom-inventory}

Overall psychopathological symptom burden was measured using the
\emph{Brief Symptom Inventory} (BSI; (\citeproc{ref-book}{Franke,
2002})), the short form of the Symptom Checklist-90-Revised (SCL-90-R;
Derogatis). The BSI consists of 53 items rated on a 5-point Likert scale
ranging from 0 (``not at all'') to 4 (``extremely''). Responses to 49
items are assigned to nine primary symptom dimensions, while four items
are evaluated separately. These symptom dimensions are summarized into
three global indices: the \emph{Global Severity Index} (GSI), reflecting
overall psychological distress; the \emph{Positive Symptom Distress
Index} (PSDI), indicating symptom intensity; and the \emph{Positive
Symptom Total} (PST), representing the number of reported symptoms.

\paragraph{Positive Mental Health
Scale.}\label{positive-mental-health-scale}

Positive mental health (PMH) was assessed with the nine-item PMH scale
(\citeproc{ref-lukat2016}{Lukat et al., 2016}). Responses are given on a
4-point Likert scale from 0 (disagree) to 3 (agree). Item scores are
summed to yield a total score ranging from 0 to 27, with higher scores
reflecting greater PMH. The scale has been validated as a unidimensional
measure with excellent internal consistency (Cronbach's α = .93), good
test--retest reliability (Pearson r = .74--.81), and evidence of scalar
invariance across samples and over time (\citeproc{ref-lukat2016}{Lukat
et al., 2016}). Furthermore, it shows strong convergent and discriminant
validity and is sensitive to therapeutic change across diverse
populations (\citeproc{ref-lukat2016}{Lukat et al., 2016}).

\paragraph{Childhood Trauma
Questionnaire.}\label{childhood-trauma-questionnaire}

Early adverse experiences were assessed using the \emph{Childhood Trauma
Questionnaire} (CTQ), a self-report measure of childhood maltreatment.
The CTQ consists of items assessing emotional, physical, and sexual
abuse, as well as emotional and physical neglect. Items are rated on a
5-point Likert scale ranging from 1 (``not at all'') to 5 (``very
often'').

Wingenfeld, Katja \& Spitzer, Carsten \& Mensebach, Christoph \& Grabe,
Hans \& Hill, Andreas \& Gast, Ursula \& Schlosser, Nicole \& Höpp,
Hella \& Beblo, Thomas \& Driessen, Martin. (2010). The German Version
of the Childhood Trauma Questionnaire (CTQ): Preliminary Psychometric
Properties. Psychotherapie, Psychosomatik, medizinische Psychologie. 60.
10.1055/s-0030-1253494.

\paragraph{Clinical Global
Impression.}\label{clinical-global-impression}

\subparagraph{Severity Scale.}\label{severity-scale}

Clinician-rated symptom severity and improvement were assessed using
the~\emph{Clinical Global Impression}~(CGI) scales.
The~\emph{CGI-Severity}~scale evaluates the clinician's global
impression of the patient's current level of mental illness, based on
their total clinical experience with this population. The item asks:
``Considering your total clinical experience with this particular
population, how mentally ill is the patient at this time?''

\subparagraph{Improvement Scale.}\label{improvement-scale}

Treatment-related change was assessed using
the~\emph{CGI-Improvement}~scale. Both patients and therapists rated
overall improvement relative to the beginning of therapy, regardless of
whether the change was attributed entirely to treatment. Patient and
therapist versions differ only in perspective but use equivalent
response formats.

\subsubsection{Global Improvement}\label{global-improvement}

Global therapy outcome was assessed using a six-point global success
rating based on two items measuring perceived benefit and satisfaction
with therapy (\citeproc{ref-michalak2003}{Michalak et al., 2003}). These
items were completed by both patients and therapists and capture a
retrospective evaluation of treatment success. The items assess (1) the
extent to which expectations toward therapy have been fulfilled and (2)
the overall perceived benefit of therapy. Responses are given on a
6-point Likert scale ranging from 1 (``on the contrary / rather
harmful'') to 6 (``completely / very helpful'').

\subsubsection{Goal Attainment Scale}\label{goal-attainment-scale}

Individualized treatment outcomes were assessed using a goal attainment
measure inspired by the Goal Attainment Scaling approach
(\citeproc{ref-kiresuk1968}{Kiresuk \& Sherman, 1968}). At the beginning
of therapy, patients and therapists collaboratively define
individualized treatment goals. During the course and at the end of
therapy, patients and therapists retrospectively evaluated the extent to
which each of the predefined goals had been achieved.

Goal attainment was rated on a standardized six-point numerical scale
ranging from deterioration relative to the initial goal state (-1 =
moved away from the goal) to full goal attainment (4 = goal achieved).
The scale reflects patients' subjective assessment of goal progress,
with intermediate categories indicating partial progress toward the
respective goal.

For each patient, an overall goal attainment score was computed as the
mean rating across all individually defined goals, representing the
average subjective level of goal progress at the end of therapy.

\subsubsection{Measurement time points}\label{measurement-time-points}

\begin{longtable}[]{@{}
  >{\raggedright\arraybackslash}p{(\linewidth - 16\tabcolsep) * \real{0.1111}}
  >{\raggedright\arraybackslash}p{(\linewidth - 16\tabcolsep) * \real{0.1111}}
  >{\raggedright\arraybackslash}p{(\linewidth - 16\tabcolsep) * \real{0.1111}}
  >{\raggedright\arraybackslash}p{(\linewidth - 16\tabcolsep) * \real{0.1111}}
  >{\raggedright\arraybackslash}p{(\linewidth - 16\tabcolsep) * \real{0.1111}}
  >{\raggedright\arraybackslash}p{(\linewidth - 16\tabcolsep) * \real{0.1111}}
  >{\raggedright\arraybackslash}p{(\linewidth - 16\tabcolsep) * \real{0.1111}}
  >{\raggedright\arraybackslash}p{(\linewidth - 16\tabcolsep) * \real{0.1111}}
  >{\raggedright\arraybackslash}p{(\linewidth - 16\tabcolsep) * \real{0.1111}}@{}}
\toprule\noalign{}
\begin{minipage}[b]{\linewidth}\raggedright
Timepoint
\end{minipage} & \begin{minipage}[b]{\linewidth}\raggedright
DU-DI
\end{minipage} & \begin{minipage}[b]{\linewidth}\raggedright
DU-Prä
\end{minipage} & \begin{minipage}[b]{\linewidth}\raggedright
KZT1-DU4
\end{minipage} & \begin{minipage}[b]{\linewidth}\raggedright
KZT1-DUPost
\end{minipage} & \begin{minipage}[b]{\linewidth}\raggedright
KZT2-DUPost
\end{minipage} & \begin{minipage}[b]{\linewidth}\raggedright
LZT1-DUPost
\end{minipage} & \begin{minipage}[b]{\linewidth}\raggedright
LZT2-DUPost
\end{minipage} & \begin{minipage}[b]{\linewidth}\raggedright
Kat6
\end{minipage} \\
\midrule\noalign{}
\endhead
\bottomrule\noalign{}
\endlastfoot
\textbf{Explanation} & Pre-therapy, 4th contact & Pre-therapy, 6th
contact & 4th therapy session & 12th therapy session & 24th therapy
session & 45th therapy session & 60th therapy session & 6 month after
therapy \\
\textbf{Diagnosis} & X & & & & & & & \\
\textbf{Demographics} & & X & & & & & & \\
\textbf{BSI} & X & & X & X & X & X & X & X \\
\textbf{BDI-II} & & X & X & X & X & X & X & X \\
\textbf{DASS-42} & & X & X & X & X & X & X & X \\
\textbf{PMH} & & & & & & & & \\
\textbf{CTQ} & & X & & & & & & \\
\textbf{CGI-S} & & X & & & & & & \\
\textbf{CGI-I} & & & & & & & & \\
\textbf{Glob-Pt} & & & & & & & & \\
\textbf{GAS} & & & & X & X & X & X & X \\
\end{longtable}

\subsection{Analysis}\label{analysis}

The Sequential Evaluation with Model Pre-registration
(\citeproc{ref-kjell_ganesan_boyd_oltmanns_rivero_feltman_carr_luft_kotov_schwartz_2024}{Kjell,
Ganesan, et al., 2024}) framework will be implemented to ensure robust
model development, mitigating overfitting and enabling unbiased
performance evaluation. Additionally, evaluating models on prospective
data will simulate real-world clinical deployment by assessing
performance on new, unseen patient data.

During the model development phase, preprocessing pipelines will be
finalized, and exploratory models developed using advanced
cross-validation techniques. Contextual embeddings derived from
pretrained LLMs will be linked to clinical outcomes using
state-of-the-art prediction models, including ridge regression
(\citeproc{ref-Hoerl1970}{Hoerl \& Kennard, 1970}), lasso regression
(\citeproc{ref-10.1111ux2fj.2517-6161.1996.tb02080.x}{Tibshirani,
1996}), and random forests (\citeproc{ref-598994}{Ho, 1995}). The final
pipelines will be pre-registered for evaluation (e.g.,
\url{https://aspredicted.org/}).

In the evaluation phase, pre-registered models will be tested on
held-out datasets, enabling unbiased performance assessments. This phase
will include detailed error analysis to identify potential biases and
limitations, as well as comparative analyses to benchmark model
performance against the HRG's models.

Furthermore, topic modeling techniques (e.g., Latent Dirichlet
Allocation; (\citeproc{ref-10.5555ux2f944919.944937}{Blei et al.,
2003}), or BERTopic;
(\citeproc{ref-grootendorst2022bertopicneuraltopicmodeling}{Grootendorst,
2022})) will be employed to explore themes in patient-generated text.
These analyses will provide valuable clinical insights into patients'
subjective experiences, including their perceived problems, impairments,
and goals. By analyzing these insights, I aim to highlight commonalities
and differences in patient narratives across diverse populations or
conditions. This process may also reveal nuanced linguistic cues that
correlate with clinical outcomes, offering a richer understanding of
patient perspectives and informing personalized care strategies.

\begin{figure}

\caption{The step-by-step process of patient narrative analysis, from
preprocessing to data evaluation.}

{\centering \includegraphics[width=1\linewidth,height=\textheight,keepaspectratio]{reports/Abbildung_Gabriel_eng.png}

}

\end{figure}%

\section{Results}\label{results}

\section{References}

\phantomsection\label{refs}
\begin{CSLReferences}{1}{0}
\bibitem[\citeproctext]{ref-Beck1996}
Beck, A. T., Steer, R. A., \& Brown, G. K. (1996). \emph{{BDI-II}: {Beck
Depression Inventory} manual} (2nd ed.). Psychological Corporation.

\bibitem[\citeproctext]{ref-10.5555ux2f944919.944937}
Blei, D. M., Ng, A. Y., \& Jordan, M. I. (2003). Latent dirichlet
allocation. \emph{Journal of Machine Learning Research}, \emph{3}(null),
993--1022.

\bibitem[\citeproctext]{ref-Bonnin2024a}
Bonnin, G., Kröber, S., \& Brachel, R. von. (2024). Die {Klassifikation}
psychischer {St{ö}rungen} mit diagnostischen {Interviews}. In T.
Teismann, P. Thoma, S. Taubner, A. Wannemüller, \& K. von Sydow (Eds.),
\emph{Klinische {Psychologie} und {Psychotherapie}: {Ein}
verfahrens{ü}bergreifendes {Lehr-} und {Lernbuch}}. Hogrefe.

\bibitem[\citeproctext]{ref-chorpita2011evidence}
Chorpita, B. F., Daleiden, E. L., Ebesutani, C., Young, J., Becker, K.
D., Nakamura, B. J., Phillips, L., Ward, A., Lynch, R., Trent, L., et
al. (2011). Evidence-based treatments for children and adolescents: {An}
updated review of indicators of efficacy and effectiveness.
\emph{Clinical Psychology: Science and Practice}, \emph{18}(2),
154--172.

\bibitem[\citeproctext]{ref-devlin-etal-2019-bert}
Devlin, J., Chang, M.-W., Lee, K., \& Toutanova, K. (2019). {BERT}:
{Pre-training} of deep bidirectional transformers for language
understanding. In J. Burstein, C. Doran, \& T. Solorio (Eds.),
\emph{Proceedings of the 2019 conference of the north {American} chapter
of the association for computational linguistics: {Human} language
technologies, volume 1 (long and short papers)} (pp. 4171--4186).
Association for Computational Linguistics.
\url{https://doi.org/10.18653/v1/N19-1423}

\bibitem[\citeproctext]{ref-book}
Franke, G. (2002). \emph{Franke, g.h. (2000). BSI. Brief symptom
inventory - deutsche version. Manual. Göttingen: beltz.}

\bibitem[\citeproctext]{ref-grootendorst2022bertopicneuraltopicmodeling}
Grootendorst, M. (2022). \emph{{BERTopic}: {Neural} topic modeling with
a class-based {TF-IDF} procedure}.
\url{https://arxiv.org/abs/2203.05794}

\bibitem[\citeproctext]{ref-598994}
Ho, T. K. (1995). Random decision forests. \emph{Proceedings of 3rd
International Conference on Document Analysis and Recognition},
\emph{1}, 278--282 vol.1.
\url{https://doi.org/10.1109/ICDAR.1995.598994}

\bibitem[\citeproctext]{ref-Hoerl1970}
Hoerl, A. E., \& Kennard, R. W. (1970). Ridge {Regression}: {Biased
Estimation} for {Nonorthogonal Problems}. \emph{Technometrics},
\emph{12}(1), 55--67.
\url{https://doi.org/10.1080/00401706.1970.10488634}

\bibitem[\citeproctext]{ref-Jensen-Doss2008}
Jensen-Doss, A., \& Weisz, J. R. (2008). Diagnostic agreement predicts
treatment process and outcomes in youth mental health clinics.
\emph{Journal of Consulting and Clinical Psychology}, \emph{76}(5),
711--722. \url{https://doi.org/10.1037/0022-006X.76.5.711}

\bibitem[\citeproctext]{ref-kiresuk1968}
Kiresuk, T. J., \& Sherman, R. E. (1968). Goal attainment scaling: A
general method for evaluating comprehensive community mental health
programs. \emph{Community Mental Health Journal}, \emph{4}(6), 443--453.
\url{https://doi.org/10.1007/BF01530764}

\bibitem[\citeproctext]{ref-kjell_ganesan_boyd_oltmanns_rivero_feltman_carr_luft_kotov_schwartz_2024}
Kjell, O. N. E., Ganesan, A. V., Boyd, R., Oltmanns, J. R., Rivero, A.,
Feltman, S., Carr, M. A., Luft, B. J., Kotov, R., \& Schwartz, H. A.
(2024). \emph{Demonstrating high validity of a new {AI-language}
assessment of {PTSD}: A sequential evaluation with model
pre-registration}. PsyArXiv. \url{https://doi.org/10.31234/osf.io/xw24e}

\bibitem[\citeproctext]{ref-kjell2019semantic}
Kjell, O. N. E., Kjell, K., Garcia, D., \& Sikström, S. (2019). Semantic
measures: {Using} natural language processing to measure, differentiate,
and describe psychological constructs. \emph{Psychological Methods},
\emph{24}(1), 92--115. \url{https://doi.org/10.1037/met0000191}

\bibitem[\citeproctext]{ref-Kjell2024}
Kjell, O. N. E., Kjell, K., \& Schwartz, H. A. (2024). Beyond rating
scales: {With} targeted evaluation, large language models are poised for
psychological assessment. \emph{Psychiatry Research}, \emph{333},
115667. \url{https://doi.org/10.1016/j.psychres.2023.115667}

\bibitem[\citeproctext]{ref-Kjell2022}
Kjell, O. N. E., Sikström, S., Kjell, K., \& Schwartz, H. A. (2022).
Natural language analyzed with {AI-based} transformers predict
traditional subjective well-being measures approaching the theoretical
upper limits in accuracy. \emph{Scientific Reports}, \emph{12}(1), 3918.
\url{https://doi.org/10.1038/s41598-022-07520-w}

\bibitem[\citeproctext]{ref-Likert1932}
Likert, R. (1932). A technique for the measurement of attitudes.
\emph{Arch. Psychol.}, \emph{140}(55).

\bibitem[\citeproctext]{ref-Lovibond1995}
Lovibond, P. F., \& Lovibond, S. H. (1995). The structure of negative
emotional states: Comparison of the Depression Anxiety Stress Scales
(DASS) with the Beck Depression and Anxiety Inventories. \emph{Behaviour
Research and Therapy}, \emph{33}(3), 335--343.
\url{https://doi.org/10.1016/0005-7967(94)00075-u}

\bibitem[\citeproctext]{ref-lukat2016}
Lukat, J., Margraf, J., Lutz, R., Veld, W. M. van der, \& Becker, E. S.
(2016). Psychometric properties of the positive mental health scale
(PMH-scale). \emph{BMC Psychology}, \emph{4}(1), 8.
\url{https://doi.org/10.1186/s40359-016-0111-x}

\bibitem[\citeproctext]{ref-Lutz2022}
Lutz, W., Schwartz, B., \& Delgadillo, J. (2022). Measurement-{Based}
and {Data-Informed Psychological Therapy}. \emph{Annual Review of
Clinical Psychology}, \emph{18}(1), 71--98.
\url{https://doi.org/10.1146/annurev-clinpsy-071720-014821}

\bibitem[\citeproctext]{ref-margraf2021}
Margraf, J., Cwik, J. C., Brachel, R. von, Suppiger, A., \& Schneider,
S. (2021). \emph{DIPS open access 1.2: Diagnostisches interview bei
psychischen störungen}. \url{https://doi.org/10.46586/rub.172.149}

\bibitem[\citeproctext]{ref-matero-etal-2019-suicide}
Matero, M., Idnani, A., Son, Y., Giorgi, S., Vu, H., Zamani, M.,
Limbachiya, P., Guntuku, S. C., \& Schwartz, H. A. (2019). Suicide risk
assessment with multi-level dual-context language and {BERT}. In K.
Niederhoffer, K. Hollingshead, P. Resnik, R. Resnik, \& K. Loveys
(Eds.), \emph{Proceedings of the sixth workshop on computational
linguistics and clinical psychology} (pp. 39--44). Association for
Computational Linguistics. \url{https://doi.org/10.18653/v1/W19-3005}

\bibitem[\citeproctext]{ref-michalak2003}
Michalak, J., Kosfelder, J., Meyer, F., \& Schulte, D. (2003). Messung
des Therapieerfolgs. \emph{Zeitschrift für Klinische Psychologie und
Psychotherapie}, \emph{32}(2), 94--103.
\url{https://doi.org/10.1026/0084-5345.32.2.94}

\bibitem[\citeproctext]{ref-mohammadi-etal-2019-clac-clpsych}
Mohammadi, E., Amini, H., \& Kosseim, L. (2019). {CLaC} at {CLPsych}
2019: {Fusion} of neural features and predicted class probabilities for
suicide risk assessment based on online posts. In K. Niederhoffer, K.
Hollingshead, P. Resnik, R. Resnik, \& K. Loveys (Eds.),
\emph{Proceedings of the sixth workshop on computational linguistics and
clinical psychology} (pp. 34--38). Association for Computational
Linguistics. \url{https://doi.org/10.18653/v1/W19-3004}

\bibitem[\citeproctext]{ref-Radford2022}
Radford, A., Kim, J. W., Xu, T., Brockman, G., McLeavey, C., \&
Sutskever, I. (2022). Robust {Speech Recognition} via {Large-Scale Weak
Supervision}. \emph{arXiv Preprint arXiv: 2212.04356}.
\url{https://arxiv.org/abs/2212.04356}

\bibitem[\citeproctext]{ref-10.1111ux2fj.2517-6161.1996.tb02080.x}
Tibshirani, R. (1996). Regression shrinkage and selection via the lasso.
\emph{Journal of the Royal Statistical Society: Series B
(Methodological)}, \emph{58}(1), 267--288.
\url{https://doi.org/10.1111/j.2517-6161.1996.tb02080.x}

\bibitem[\citeproctext]{ref-Vaswani2017}
Vaswani, A., Shazeer, N., Parmar, N., Uszkoreit, J., Jones, L., Gomez,
A. N., Kaiser, L., \& Polosukhin, I. (2017). Attention is {All} you
{Need}. \emph{Advances in {Neural Information Processing Systems}},
\emph{30}.

\bibitem[\citeproctext]{ref-wampold2015great}
Wampold, B. E., \& Imel, Z. E. (2015). \emph{The great psychotherapy
debate: {The} evidence for what makes psychotherapy work}. Routledge.

\bibitem[\citeproctext]{ref-WorldHealthOrganization2017}
World Health Organization. (2017). \emph{Depression and other common
mental disorders: {Global} health estimates}.

\bibitem[\citeproctext]{ref-zirikly-etal-2019-clpsych}
Zirikly, A., Resnik, P., Uzuner, Ö., \& Hollingshead, K. (2019).
{CLPsych} 2019 shared task: {Predicting} the degree of suicide risk in
{Reddit} posts. In K. Niederhoffer, K. Hollingshead, P. Resnik, R.
Resnik, \& K. Loveys (Eds.), \emph{Proceedings of the sixth workshop on
computational linguistics and clinical psychology} (pp. 24--33).
Association for Computational Linguistics.
\url{https://doi.org/10.18653/v1/W19-3003}

\end{CSLReferences}






\end{document}
